\section{Conclusion}

In this paper we discussed the results of an exact algorithm for the computation of the $k$ most central nodes in a graph, according to closeness centrality. We saw that with the introduction of a lower bound, the real word performance are way better than a brute force algorithm that compute all the \texttt{BFS}. \s

\nd Since there were no server with dozens of threads and hundreds of Gigs of RAM, every idea has been implement knowing that it needed to run fine on a regolar laptop. This condition lead to interesting implementations for the filtering on the raw data. \s

\nd We have seen two different case studies, both based on the IMDb network. For each of them we had to find a way to filter the data without loosing accuracy on the results. We saw that with an harder filtering, we gain a lot of performance, but the results showed an increasing discrepancy from the reality. Analyzing those tests made, we have been able to find, for both graphs, a balance that gives accuracy and performance at the same time.

\s \nd This work is heavily based on \cite{DBLP:journals/corr/BergaminiBCMM17}. Even if this article use a more complex and complete approach, the results on the IMDb case study are almost identical. They worked with snapshot, analyzing single time periods, so there are some inevitable discrepancies. Despite that, most of the top-$k$ actors are the same and the closeness centrality values are very similar. We can use this comparison to attest the truthfulness and efficiency of the algorithm presented in this paper.

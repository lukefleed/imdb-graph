\section{An overview of the code}
The algorithm implement is multi-threaded and written in C\texttt{++}

\subsection{Data structures}
In this case we are working with two simple \texttt{struct} for the classes \emph{Film} and \emph{Actor}

\lstinputlisting[language=c++]{code/struct.cpp}
\s
\nd Then we need two dictionaries build like this

\lstinputlisting[language=c++]{code/map.cpp}
\s
\nd We are considering the files \texttt{Attori.txt} and \texttt{FilmFiltrati.txt}, we don't need the relations one for now. Once that we have read this two files, we loop on each one brutally filling the two dictionaries created before. If a line is empty, we skip it. We are using a try and catch approach. Even if the good practice is to use it only for a specific error, since we are outputting everything on the terminal it makes sense to \emph{catch} any error.

\lstinputlisting[language=c++]{code/data.cpp}
\s

Now we can use the file \texttt{Relazioni.txt}. As before, we loop on all the elements of this file, creating the variables

\begin{itemize}
    \item \texttt{id\textunderscore film}: index key of each movie
    \item \texttt{id\textunderscore attore}: index key of each actor
\end{itemize}

\nd If they both exists, we update the list of indices of movies that the actor/actresses played in. In the same way, we updated the list of indices of actors/actresses that played in the movies with that id.

\lstinputlisting[language=c++]{code/graph.cpp}
\s
Now that we have defined how to build this graph, we have to implement the algorithm what will return the top-k central elements. \s

\nd The code can be found here: \url{https://github.com/lukefleed/imdb-graph}
\s
\begin{center}
    \qrcode[height=1in]{https://github.com/lukefleed/imdb-graph}
\end{center}

Understanding and investigating social structures is essential in the modern world. Through the use of networks and graph theory we can find the most central elements in a community. In particolar, given a connected graph $G=(V,E)$, the closeness centrality of a vertex $v$ is defined as $ \frac{n-1}{\sum_{w \in V} d(v,w)}$. This measure can be seen as the efficiency of a node to pass information through all the other nodes in the graph. In this paper we will discuss and algorithm and its result for finding the top-k most central elements in web-scale graphs. As a case study, we are going to use the IMDB collaboration network, building two completely different graphs and analyzing their proprieties.

% Given a connected graph $G=(V,E)$, the closeness centrality of a vertex $v$ is defined as $ \frac{n-1}{\sum_{w \in V} d(v,w)}$. This measure is widely used in the analysis of real-world complex networks, and the problem of selecting the $k$ most central vertices has been deeply analysed in the last decade. However, this problem is computationally not easy, especially for large networks. I propose an algorithm for selecting the $k$ most central nodes in a graph: I experimentally show that this algorithm improves significantly both the textbook algorithm, which is based on computing the distance between all pairs of vertices, and the state of the art. Finally, as a case study, I compute the $10$ most central actors in the IMDB collaboration network, where two actors are linked if they played together in a movie.

% Da cambiare le parole, preso dal paper
